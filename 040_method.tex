% !TeX spellcheck = ru_RU
% !TEX root = vkr.tex

\section{Метод}
В этом разделе будет показана идея оптимизации \enquote{слияние ядер} для деревьев квадрантов в деталях.
\subsection{Слияние ядер}
Основная идея данной техники заключается в том, чтобы за один проход по какой-либо структуре данных выполнять несколько независимых процессов в один, это позволяет избегать промежуточных структур данных, сразу генерируя результат. Применительно к линейной алгебре, если имеется несколько матриц $A = (a_{i, j}), B = (b_{i, j}), C = (c_{i, j})$ с одинаковыми размерностями $m \times n$ и поэлементная бинарная  операция $*$, то можно записать применение этой операции последовательно к матрицам как $\mathbf{map2} \ (*) \ (\mathbf{map2}\ (*)\ A \ B) \ C$, таким образом будет подсчитана промежуточная матрица $\mathbf{map2}\ (*)\ A \ B = (a_{i, j} * b_{i, j})$, иначе можно ввести функцию $\mathbf{map3}$, которая будет семантически совпадать с конструкцией $\mathbf{map2}\ (*) \ (\mathbf{map2}\ (*)\ A' \ B') \ C'$, но написать её таким образом, чтобы на каждом шаге подсчитывалось непосредственно $a_{i, j} * b_{i, j} * c_{i, j}$.
