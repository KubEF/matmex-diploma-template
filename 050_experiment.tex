% !TeX spellcheck = ru_RU
% !TEX root = vkr.tex

\section{Эксперимент}
\subsection{Цель эксперимента}
Целью эксперимента является сравнение оптимизированных версий функций с их неоптимизированными аналогами по времени исполнения.
\subsection{Условия эксперимента}
Разреженные матрицы были взяты из SuiteSparse Matrix Collection\footnote{SuiteSparse Matrix Collection: \url{http://sparse.tamu.edu/}. (дата обращения:   \DTMdate{2023-12-09}).} и представлены в таблице
\ref{used_matrixes}, где размер уже приведён к соответствующим степеням двойки для корректного построения дерева квадрантов.
%поправить и записать нормально размер
\begin{table}[h!]
    \rowcolors{2}{black!2}{black!10}
    \def\arraystretch{1.1}  % Растяжение строк в таблицах
    \setlength\tabcolsep{0.2em}
    \centering
    % \resizebox{\linewidth}{!}{%
    \caption{Использованные матрицы}
    \begin{tabular}[C]{l|
            c
            |r|r}
        \toprule
        Название &\multicolumn{1}{r|}{Размер}& Ненулевые элементы&\% заполненности  \\ \midrule
        meg4        ($Q_1$)& {$2^{13}$}\times{$2^{13}$} & 26324& 0,0766 \\
        aircraft    ($Q_2$)& {$2^{13}$}\times{$2^{13}$} & 20267 & 0,0718 \\
        fd12        ($Q_3$)& {$2^{13}$}\times{$2^{13}$} & 28462 & 0,0506 \\
        mc2depi     ($Q_4$)& {$2^{20}$}\times{$2^{20}$} & 2100225 & 0,0007 \\
        Hardesty2   ($Q_5$)& {$2^{20}$}\times{$2^{20}$} & 4020731 & 0,0014\\
        ecology1    ($Q_6$)& {$2^{20}$}\times{$2^{20}$} & 2998000 & 0,0002 \\
        webbase-1M  ($Q_7$)& {$2^{20}$}\times{$2^{20}$} & 3105536 & 0,0003\\
        \bottomrule
    \end{tabular}
    \label{used_matrixes}
\end{table}

Ниже представлены характеристики оборудования, на котором производились измерения.
\begin{verbatim}
    Operating System: Debian, unstable
    \end{verbatim}

\subsubsection*{CPU}
\begin{verbatim}
    Architecture:       x86_64
    Model name:         Intel(R) Core(TM) i7-4790 CPU @ 3.60
    \end{verbatim}

\subsubsection*{RAM}
\begin{verbatim}
    Total (Gi): 32
\end{verbatim}
Для измерений использовалась библиотека criterion\footnote{Criterion: \url{https://github.com/haskell/criterion}. (Дата обращения: \DTMdate{2023-12-10})}.

Были произведены замеры работы следующих функций:
\begin{itemize}
    \item measurement1 = $zipWithBinFunc4\ (+) \ Q_6 \ Q_6 \ Q_6\ Q_6$
    \item measurement2 = $zipWithBinFunc4\ (+) \ Q_5 \ Q_7 \ Q_5\ Q_6$
    \item measurement3 = $zipWith4Funcs\ (+) \ (*)\ (revDiv) \ Q_5 \ Q_7 \ Q_5\ Q_6$
    \item measurement4 = $map2\ revDiv\ (Q_3 \times Q_1)\ Q_2 $
\end{itemize}
Где $(+),\ (*), \ (revDiv) $ ~--- это операции сложения, умножения и обратного деления ($revDiv \ a \ b = b / a $) над вещественными числами соответственно, $\times$~--- операция умножения матриц над вещественными числами.
\begin{itemize}
    \item $map2$~--- функция, применяющая данную бинарную операцию элементам первой матрицы к элементам второй\\($A_{m\times n} = (a_{i, j}), \ B_{m\times n} = (b_{i, j}), \quad map2 \ (+)\ A \ B = (a_{i, j} + b_{i, j})$);
    \item $zipWithBinFunc4$~--- функция, эквивалентная последовательному применению $map2$ относительно одной и той же функции к данным четырём матрицам;
    \item $zipWith4Funcs$~--- функция, эквивалентная последовательному применению $map2$ относительно соответственно трёх переданных бинарных функций к четырём переданным матрицам.
\end{itemize}
\subsection{Результаты}
Результаты измерений представлены в таблице \ref{results}, где указано среднее время $\pm$ стандартное отклонение в секундах.
Оптимизированная версия на \Th{}, обычная версия на \Haskell{}.

\begin{table}[h!]
    \rowcolors{2}{black!2}{black!10}
    \def\arraystretch{1.1}  % Растяжение строк в таблицах
    \setlength\tabcolsep{0.2em}
    \centering
    % \resizebox{\linewidth}{!}{%
    \caption{Результаты измерений}
    \begin{tabular}[C]{l|
            *2{S
                        [table-figures-uncertainty=3, separate-uncertainty=true, table-align-uncertainty=true,
                            table-figures-integer=2, table-figures-decimal=3, round-precision=3,
                            table-number-alignment=center]
                }
        }
        \toprule
        Measurement num& \multicolumn{1}{r}{\Th}& \multicolumn{1}{r}{\Haskell} \\ \midrule
        measurement1 & 1.05 \pm 0.08 & 2.53 \pm0.12   \\ \midrule
        measurement2 & 4.010 \pm 0.430 & 7.37 \pm0.001  \\ \midrule
        measurement3 & 2.28 \pm 0.001 & 4.57 \pm  0.447  \\ \midrule
        measurement4 & 74.2 \pm 0.46  & 76.7 \pm  0.07   \\
        \bottomrule
    \end{tabular}
    \label{results}
\end{table}

Анализируя полученные данные, можно заключить, что оптимизация позволила значительно, почти в 2 раза, сократить время работы программы для поэлементных операций.
