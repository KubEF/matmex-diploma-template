% !TeX spellcheck = ru_RU
% !TEX root = vkr.tex

\section{Эксперимент}
\subsection{Цель эксперимента}
Целью эксперимента является сравнение оптимизированных версий функций с их не оптимизированными аналогами.
\subsection{Условия эксперимента}
Разряженные матрицы были взяты из SuiteSparse Matrix Collection\footnote{SuiteSparse Matrix Collection: \url{http://sparse.tamu.edu/}. (дата обращения:   \DTMdate{2023-12-09}).} и представлены в таблице
\ref{used_matrixes}

\begin{table}[h!]
    \rowcolors{2}{black!2}{black!10}
    \def\arraystretch{1.1}  % Растяжение строк в таблицах
    \setlength\tabcolsep{0.2em}
    \centering
    % \resizebox{\linewidth}{!}{%
    \caption{Использованные матрицы}
    \begin{tabular}[C]{l|r|r}
        \toprule
        Название &Размер& Ненулевые элементы \\ \midrule
        1138\_bus   ($Q_1$) & 1295044  & 2596 \\
        ex31        ($Q_2$)& 15280281 & 115357 \\
        meg4        ($Q_3$)& 34339600 & 26324 \\
        lp\_cre\_b  ($Q_4$)& 744217776 & 260785 \\
        mc2depi     ($Q_5$)& 276491930625 & 2100225 \\
        Hardesty2   ($Q_6$)& 282359789145 & 4020731 \\
        GL7d16      ($Q_7$)& 439608168408 & 14488881 \\
        ecology1    ($Q_8$)& 1000000000000 & 2998000 \\
        webbase-1M  ($Q_9$)& 1000010000025  & 3105536 \\
        \bottomrule
    \end{tabular}
    \label{used_matrixes}
\end{table}

{\small\textit{Здесь будут характеристики билдербоя}}. Для измерений использовалась библиотека criterion\footnote{Criterion: \url{https://github.com/haskell/criterion}. (Дата обращения: \DTMdate{2023-12-10})}.

Были произведены замеры работы следующих функций:
\begin{itemize}
    \item measurement1 = $zipWithBinFunc4\ (+) \ Q_1 \ Q_1 \ Q_1\ Q_1$
    \item measurement2 = $zipWithBinFunc4\ (+) \ Q_6 \ Q_7 \ Q_8\ Q_9$
    \item measurement3 = $zipWith4Funcs\ (+) \ (*)\ (\cdot) \ Q_8 \ Q_9 \ Q_7\ Q_7$
    \item measurement4 = $map2\ (\cdot)\ (Q_1 \times Q_2)\ Q_1 $
\end{itemize}
Где $(+),\ (*), \ (\cdot), \ (-) $ ~--- это операции сложения, умножения конкатенации и разности над вещественными числами соответственно, $\times$~--- операция умножения матриц над вещественными числами.
\begin{itemize}
    \item $map2$~--- функция, применяющая данную бинарную операцию элементам первой матрицы к элементам второй\\($A_{m\times n} = (a_{i, j}), \ B_{m\times n} = (b_{i, j}), \quad map2 \ (+)\ A \ B = (a_{i, j} + b_{i, j})$);
    \item $zipWithBinFunc4$~--- функция, эквивалентная последовательному применению $map2$ относительно одной и той же функции к данным четырём матрицам;
    \item $zipWith4Funcs$~--- функция, эквивалентная последовательному применению $map2$ относительно соответственно трёх переданных бинарных функций к четырём переданным матрицам.
\end{itemize}
\subsection{Результаты}
Результаты измерений представлены в таблице \ref{results}, где указано среднее время $\pm$ стандартное отклонение в секундах.Оптимизированная версия на \Th{} (TH), обычная версия на \Haskell{} (RH)

\textit{Здесь будет табличка с измерениями, пока не успеваю, но там будет видно, что оптимизированная версия выигрывает у неоптимизированной в поэлементных операциях, но не даёт значительного выигрыша при умножении}
\begin{table}[h!]
    \rowcolors{2}{black!2}{black!10}
    \def\arraystretch{1.1}  % Растяжение строк в таблицах
    \setlength\tabcolsep{0.2em}
    \centering
    % \resizebox{\linewidth}{!}{%
    \caption{Результаты измерений}
    \begin{tabular}[C]{l|r|r}
        \toprule
        Measurement num& TH& RH \\ \midrule
        \bottomrule
    \end{tabular}
    \label{results}
\end{table}


Анализируя полученные данные, можно заключить, что оптимизация позволила значительно сократить время работы программы для поэлементных операций.
