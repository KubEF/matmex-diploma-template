% !TeX spellcheck = ru_RU
% !TEX root = vkr.tex

\section*{Введение}
% Использование линейной алгебры встречается в различных областях: от машинного обучения [ссылочка на статью] до теории графов [ссылочка на статью].
% \begin{enumerate}
%     \item Откуда и почему появляются разряженные матрицы.
%     \item Почему хочется улучшать производительность.
%     \item Почему выбран для этих целей template haskell.
%     \item Что такое kernel fusion.
%     \item Про template haskell
% \end{enumerate}
При работе с большими объёмами данных возникает множество проблем, одна из них~--- промежуточные структуры данных, которые никак не используются в результате, могут давать большую нагрузку на память и замедлять работу программы.
Более или менее частные решения этой проблемы существуют: \textit{stream fusion}~\cite{stream_fusion}, дефорестация~\cite{deforestation}, суперкомпиляция~\cite{supercompiler}, дистилляция~\cite{distillation}.
Данные подходы имеют смысл, однако некоторые из них применимы только в крайне специфических областях, например, \textsc{Tensorflow}~\cite{tensorflow} для работы с тензорами, которые нужны для алгоритмов машинного обучения.
Или наоборот, крайне общие, очень теоретизированные и плохо \enquote{реально} изученные, например, дистилляция.

Таким образом, возникает идея изучения какого-то из наиболее общих методов в каких-то частных случаях для более лёгкого анализа.
Например, разреженная линейная алгебра имеет приложения  в теории графов~\cite{kepner2011graph} и машинном обучении~\cite{8091098} и при этом является крайне удобным объектом для изучения в силу большого количества работ и элегантной математической структуры.
Что касается метода, то слияние ядер~\cite{5724850} позволяет уменьшить количество промежуточных структур данных посредством одновременной обработки узлов.
Поэтому реализация данной техники в контексте разреженной линейной алгебры видится удачной.

Так как большинство литературы о новейших из этих методов используют модельные языки, имеющие функциональную природу, было решено писать на языке \Haskell{}, а в качестве частного случая рассматривать матрицы, представленные в виде деревьев квадрантов, крайне удобные для обработки в данной парадигме.
